\subsection[X86 ISA Improvements]{X86 ISA Improvements\footnote{by Nilay Vaish}}

The x86 or x86-64 ISA is one of the most popular ISAs for desktop, server, and high-performance compute systems.
Thus, there has been significant effort to improve gem5's modeling of this ISA.
This section presents a subset of the changes to improve the x86 ISA.
There are many other improvements large and small that generally have improved the fidelity of x86 modeling.

In out-of-order CPUs (e.g., gem5's O3CPU), instructions whose dependencies have been satisfied are allowed to execute even if there are instructions earlier in the stream waiting for their operands.
The flag register used in the x86 ISA complicates this as almost every instruction both reads and writes this register making them all dependent on one another.
Maintaining a single flag register can introduce dependencies that need not exist.
We now maintain multiple flag registers for holding subsets of flag bits to reduce the dependencies.
This prevents unnecessary serialization, unlocking a significant amount of instruction-level parallelism.

Memory consistency models decide the amount of parallelism available in a memory system, while correctly executing a program.
The x86 architecture is based on the Total Store Order (TSO) memory model~\cite{coherenceprimer}.
We added support for TSO to gem5 for the x86 architecture.
This meant ensuring that a later load from a thread can bypass earlier loads/stores, but stores from the same thread are always executed in order.
The out-of-order CPU model in gem5 has been improved to implement both TSO and more relaxed consistency models (e.g., those in the RISC-V~\ref{sec:riscv} and Arm~\ref{sec:arm} architectures).
